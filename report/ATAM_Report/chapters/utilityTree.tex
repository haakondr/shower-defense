Table \ref{utilityTable} describe the utility tree generated during the ATAM exercise. The scenarios are ranked in regards to difficulty (D) and priority (P), which should give an indication of the importance and complexity of implementing the scenario.

\begin{table}[h]
  \begin{tabularx}{350pt}{|l | l | l | l| X|}
	\hline	
	\textbf{Attribute} & \textbf{P} & \textbf{D} & \textbf{Scenario} & \textbf{Details}\\
	\hline
	Modifiability & H & M & Helicopter & The developer adds helicopters for the soldiers \\
	\hline
	Modifiability & H & L & Game speed & The developer change the speed of the game \\
	\hline
	Modifiability & M & L & Score & The developer change the scoring system \\
	\hline
	Testability & H & H & Testing scenario & Test personnel wants to test input and observe output of the class \\
	\hline
  \end{tabularx}
  \caption{Utility Table}
  \label{utilityTable}
\end{table}

We picked the "The developer adds helicopter" scenario because this scenario concerns some of the main functionality of the game, and we regarded it as the most important scenario. The "change the speed of the game" scenario was picked because it is needed to change the difficulty of the game, and thus providing a longer gameplay lifetime for the user. The "developer change the scoring system" scenario was picked because it also would improve the gameplay lifetime. The final scenario "test personnel wants to test input and observe output of the class", was chosen because it would improve the testability of the game, and thus help in ensuring its quality.

