The decisions regarding our choice of architecture is mostly explained in the previous sections, however we will give a brief summary of the reasons behind the architectural decisions.

The architetural tactics employed in relation to the selected quality attributes revolves mostly around the code structure. Prevention of ripple effects, generalizing modules, the SRP-principle, are all tactics providing less complexitiy to the code. Development of a game with graphical animation can often times provide relatively complex code, something which can provide several problems especially considering our main quality attributes. By employing the selected tactics we always look for simple solutions with a clean structure, thus lowering the complexity.

As for the architectural patterns, it seemed that the MVC would be a reasonable choice as an architectural pattern since it aims to seperate the business logic from the user interface. This is a solution which also benefits the previous mentioned complexity-issues -- separating complex logic into designated areas. Even though it became difficult for us to fully implement this pattern, an issue described in the next section, we can see the benefits from it.    
