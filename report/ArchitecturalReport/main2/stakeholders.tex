\label{stakeholders}

\subsection{Course staff}
The concerns of the course staff would be how easy it is to compile-and-run the final product, and how good the documentation of the code
is. These are concerns for them because they will evaluate our product according to these criteria.

\subsection{Android users}
The users of the game are Android users who wants to play the game. One of the concerns for this stakeholder is how easy the game is to run and
play. This means that the instructions on how to play the game should be easily
available and easy to understand. The stakeholder is also concerned with how good the usability for the game is. That is, how easy the game is to get familiar with and the sense of control it provides.

\subsection{Programmers }
The programmers concerns are mainly with the implementation of the system. They want the architecture to be in such a way that further implementation and improvement is as easy to perform as possible. By using patterns, new developers can recognize these concepts and thus understand the code more easily. Good documentation of the code is also a valid concern of the programmers.

\subsection{Testers} 
Testability is the main concern for the testers. They want the system to be as easy to test as possible, to ensure that all features can be checked properly. Good testability will result in less work for the testers, and thus, lower costs.

\subsection{ATAM evaluation team}
The ATAM evaluation team is responsible for evaluating the architecture of the system, by assessing if the architecture satisfy the quality goals set. Documentation of architecture is the main concern for the evaluation team. This is due to how easy it is to comprehend the architecture design for the evaluation team, and how the team can rate the architecture in regard to the quality goals.
